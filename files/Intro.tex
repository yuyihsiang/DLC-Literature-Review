\section{Introduction}

As the development of wave energy converters (WECs) has increased in detail and scope, increasing attention is now being given to the prediction of design responses and loads for devices to allow for thorough and efficient WEC design. 
Extreme conditions often drive the maximum design loads and fatigue requirements. Accordingly, the prediction of extreme responses and loads is a critical step in the device design process. 
The inability to accurately determine device responses and loads in extreme seas and a lack of confidence in design methods likely result in over-inflated fabrication and maintenance costs.
The need to both ensure device survival and minimize safety factors (and therefore unnecessary structural costs) has thus created an increased incentive to better understand design responses and loads for WECs.

In 2014, a workshop was hosted by researchers from Sandia National Laboratory and National Renewlable Energy Labaratory, as part of the extreme condition modeling (ECM) project, to gather WEC researchers and developers with the goal of better understanding the state of the art in WEC design response analysis (\cite{Coe2014}). 
The workshop attendees highlighted that while design response analysis is widely understood as a critical step in the design process for WECs, additional development in this area would be crucial for the success of the industry. 
Both device modeling tools and methods for determining environmental loads were identified as key areas in need of further development. 
Additionally, a need for more publicly available experimental datasets for validation of design processes and numerical models was noted. 
The workshop attendees also agreed that there was a need for additional guidelines, best-practices documents and standards to normalize the design process of the WECs and increase investor confidence. 
The recommendations from this workshop highlight the need for improvements to components within the WEC design process, as well as increased understanding and confidence in the WEC design process as a whole.

An ECM framework was developed based on the recommendation from the workshop and was presented by \cite{Yu2015}. One of the short tern objectives for the ECM project is to develop a best-practices document that describes the further developed framework for WEC design response and load analyses, including guidelines and examples on design load case (DLC) studies. 
To provide a stepping stone for the best-practices document, a thorough review was conducted on the current status of any existing WEC design load guidelines, as well as the guidelines developed for offshore oil and gas, naval architecture and offshore wind. Details of the review are documented by \cite{Coe2017, VanRij2018}. This report will provide an overview of the two studies performed by the ECM team and will be divided into two parts, accordingly. The first part of the report will provide an overview of WEC reliability, survival and design practices (\cite{Coe2017}). The second part will focus on the current state of art of existing approaches on DLC analysis (\cite{VanRij2018}).